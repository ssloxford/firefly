\begin{abstract}
Data from Earth-observing satellites has become crucial in private enterprises, research applications, and in coordinating national responses to events such as forest fires.
These purposes are supported by datasets derived from a variety of satellites, some of which were launched decades ago.
Since then, attitudes to securing the physical layer have changed, opening the door for modern adversaries to overshadow the signal with commercially available radio equipment.

In this paper, we demonstrate how satellite images from widely-used satellites can be manipulated by an attacker operating off-the-shelf radio equipment.
Taking NASA's live forest fire detection system as a case study, we demonstrate that the attacker can arbitrarily manipulate fires in the derived dataset to trigger false emergency response or mislead crisis analysis, and achieve denial of service in the processing software.
We provide a simulation framework to identify and measure the key properties required for attack success.
We conclude with an evaluation of existing countermeasures to detect and defend against such an attacker.

% Through radio simulation, we show that the required signal overshadowing can be achieved by a terrestrial attacker despite the highly directional nature of the dish.
\end{abstract}
