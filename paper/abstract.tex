\begin{abstract}
Data from Earth-observing satellites has become crucial in private enterprises, research applications, and in coordinating national responses to events such as forest fires.
These purposes are supported by datasets derived from a variety of satellites, some of which were launched decades ago.
Since then, attitudes to securing the physical layer have changed, opening the door for modern adversaries to overshadow the signal with commercially available radio equipment.

In this paper, we consider how this leads to poisoning satellite-derived datasets and exploiting the processing systems themselves.
As a case study, we consider NASA's live forest fire detection system, which is used by the fire agencies of over 90 countries \textit{TODO: must cite later}.
We demonstrate that the attacker can arbitrarily manipulate fires in the derived dataset to trigger false emergency response or mislead crisis analysis, and achieve denial of service and code execution in the processing software.
Through radio simulation, we show that the required signal overshadowing can be achieved by a terrestrial attacker despite the highly directional nature of the dish.
This raises concerns that any dataset derived from these satellites could be compromised by similar means.
\end{abstract}
