\section{Threat Model}\label{sec:threat-model}

The goal of the adversary is to cause disruption to downlink processing systems by emitting conterfeit signals in the vicinity of the receiver.
This disruption can either seek to affect the execution of specific processing pipeline stages, or the resulting satellite-derived datasets by:

% In light of what we've shown about SDS being use din critical applications, where the attacker's goal is to affect the downlink systems.

\noindent\textit{Affecting the satellite-derived datasets}
\begin{itemize}
    \item Inject false data to mislead people -- for example, to disrupt automated systems for forest fire and other anomaly detection;
    \item Mask important data to deny people information -- for instance, to hide approaching natural disasters.
\end{itemize}

\noindent\textit{Exploiting or disrupting downlink processing stages}
\begin{itemize}
    \item Achieve denial of service -- for example, by causing processing pipeline stages to crash or output malformed data;
    \item Execute arbitrary code -- for instance, by exploiting boundaries between processing applications that have access to the shell.
\end{itemize}

\textbf{TODO: rewrite this paragraph - it's a mixture of ultimate goals and effects to the dataset}
By achieving these, an attacker could aim to reroute traffic around ficticious fires, destroy or manipulate recent or historical archived data, prevent new data from being processed and saved, or hide malicious payloads in a long-term storage database, to be triggered by a later processing step.

We assume the attacker has access to commercially available off-the-shelf radio equipment such as Software Defined Radios (SDRs), amplifiers, and antennas, in order to emit signals in the X-band frequency range used by the EOS downlink.
We also assume the attacker is able to maintain a presence in the vicinity of the receiver, in order to inject signals of a sufficient strength that they will be picked up by the receiver.
Crucially, since the receiving satellite dish is highly directional and the attacker is likely ground-based, we do not assume that the attacker can emit sigals within the beam of the receiver.
Instead, they must rely on the signal being sufficiently strong that the ground station picks it up regardless of dish orientation. % TODO: diagram of dish orientation and the attacker's location?

In order to define realistic price constraints, we consider off-the-shelf radio equipment intended to be used in cubesats.
We estimate the overall equipment cost to be approximately $30,000$ EUR -- such a budget covers the cost of a suitable SDR (\cite{limeSdr,limeCompanion}, $598$ USD), a transmitter capable of amplifying and transmitting signals in the X-band (\cite{endurosat:xbandtransmitter}, $22,800$ EUR), and a compatible antenna (\cite{endurosat:xbandantenna}, $6,400$ EUR).
%One cubesat manufacturer, EnduroSat, offers a system capable of encoding and transmitting signals in the X-band for €22,800 and a compatible antenna for €6,400~\cite{endurosat:xbandtransmitter,endurosat:xbandantenna}.
These prices place the attacks within the budget range of a motivated hobbyist, opening the EOS systems up to attacks by a wide range of parties.

We also take into account other mechanisms by which the attacker could carry out the intended goals, beyond the scope of the physical layer.
The mailing list for NASA's Direct Readout Laboratory (DRL) provides an attractive entry point -- when a downlink transmission is missed by NASA's ground stations or by another organization, it is routine to use the mailing list to ask someone to send over their own copy of the data~\textbf{TODO cite}. % Is such a citation possible?
Our attacker can take advantage of the trust inherent to this system's operation and send over malicious frames, which may processed and entered into the DRL's database.
Of course, it is more difficult to carry out this attack at a specific time due to the ``request/response'' nature of the exchange, but it may still be possible to meddle with archived data or attack the processing systems via this method.
