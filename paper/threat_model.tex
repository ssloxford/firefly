\section{Threat Model}\label{sec:threat-model}

The goal of the adversary is to cause disruption to the EOS image processing pipeline by emitting conterfeit signals in the vicinity of the receiver.
This disruption can take a number of forms:
\begin{itemize}
    \item Inject false data to mislead people -- for example, to disrupt automated systems for forest fire and other anomaly detection,
    \item Mask important data to deny people information -- for instance, to hide approaching natural disasters,
    \item Prevent new data from being processed and saved by the receiver,
    \item Destroy or alter recent or historical data,
    \item Hide malicious payloads in a long-term storage database, to be triggered by a later processing step.
\end{itemize}
%Depending on the scenario, this disruption could take the form of denial of service to the software or hardware components, modification of recently received or historical image data, or destruction of old data.

We assume the attacker has access to commercially available off-the-shelf radio equipment such as Software Defined Radios (SDRs), amplifiers, and antennas, in order to emit signals in the X-band frequency range used by the EOS downlink.
We also assume the attacker is able to maintain a presence in the vicinity of the receiver, in order to inject signals of a sufficient strength that they will be picked up by the receiver.
Crucially, we do not necessarily assume the attacker can emit signals within the beam of the receiving ground station, instead relying on the signal being sufficiently strong that the ground station picks it up regardless of dish orientation.

In order to define realistic price constraints, we consider off-the-shelf radio equipment intended to be used in cubesats.
One cubesat manufacturer, EnduroSat, offers a system capable of encoding and transmitting signals in the X-band for €22,800 and a compatible antenna for €6,400~\cite{endurosat:xbandtransmitter,endurosat:xbandantenna}.
This system has an integrated encoder and is thus not capable of transmitting attacker-crafted sequences of symbols, but it provides us with an understanding of how much similar equipment would cost.
These prices place the attacks within the budget range of a motivated hobbyist, opening the EOS systems up to attacks by a wide range of parties.

We also need to take into account other mechanisms by which the attacker could carry out the intended goals, beyond the scope of the physical layer.
The mailing list for NASA's Direct Readout Laboratory (DRL) provides an attractive entry point -- when a downlink transmission is missed by NASA's ground stations or by another organization, it is routine to use the mailing list to ask someone to send over their own copy of the data~\textbf{TODO cite}.
Our attacker can take advantage of the trust inherent to this system's operation and send over malicious frames, which will be processed without question and entered into the DRL's database.
Of course, it is more difficult to carry out this attack at a specific time due to the ``request/response'' nature of the exchange, but it is still possible to meddle with archived data or attack the processing systems via this method.

Finally, we must consider the sphere of influence of any attacks on the processing systems.
When NASA receives and processes a transmission from a satellite in the EOS fleet, the resulting data is used in a wide range of applications.
We focus on forest fire detection in this paper, but the attacks we describe can easily be adapted to affect other systems connected to the same data streams.
In Section~\ref{sec:background} we review services provided by relying on satellite-derived data; all the services relying on EOS data to provide their services are vulnerable to attacks.

The same is true of any other organisation which processes EOS data, although usually to a lesser extent.
If instead of NASA the attacker targets this organisation's ground station and attached processing system, anything relying on the output data is once again vulnerable.
The distributed nature of ground stations serves to make attacks a little more difficult, but it is still the case that attacking a single ground station affects all datasets derived from data received by that ground station.

Fundamentally, the issue stems from implicit trust of the data source, allowing vulnerabilities to one service to affect any other services further down the processing chain.
\textbf{TODO finish, possibly move further on in the paper}

%The goal of the adversary is to stealthily spoof the image information decoded by a single receiver station by emitting a counterfeit signal in the vicinity of the receiver.  Depending on the scenario, the attacker may want to inject similar-looking images with slight modifications to disrupt automated systems for forest fire and other anomaly detection.  In other cases, the attacker may wish to inject adversarial examples or degrade image quality to harm the research efforts of those using data derived from this source.

%We assume that the attacker has access to off-the-shelf equipment, such as software-defined radios, amplifiers, and antennas.  We also assume that the attacker is able to maintain presence in the vicinity of the images they want to spoof, enabling them to receive the direct broadcast signal for that area [TODO: include roughly how precisely they need to be in the location].  They must also have a presence near the targeted receiver station, and be able to transmit the information they decode between the locations, using systems such as a wired, mobile, or even satellite internet connection.

%We assume that the receiver station has strong waveform and timing analysis on its input signals, utilising state-of-the-art spoofing detection mechanisms.  Although currently few receiver stations perform this analysis, suspicions about the origin of the data may still lead to retrospective analysis.  The attacker must therefore remain stealthy, and manipulate the signal only in ways that still resemble other legitimate signals.

%Finally, the attacker is limited by the physical properties of the radio signal: they are unable to perform the required signal cancellation simply through analogue means, due to the highly limited distance constrains that this requires.  Instead, the attacker must seek to cancel the signal from a distance, requiring that they first receive and process the direct broadcast signal.
