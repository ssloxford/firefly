\section{Conclusion}

Having demonstrated the impact of signal injection attacks on systems which assume input data to be safe, a discussion remains to be had on how this situation was reached in the first place.
Fundamentally, the assumption that all input data is safe led to a lack of safety checks and input sanitization in the pipeline, allowing any data to be processed, including that which was crafted by an attacker.
This is compounded by a lack of security in bundled dependencies -- the system contains many redundant dependencies, including ones years out of date.
This opens the system up to a huge number of known vulnerabilities which could be mitigated simply by using the latest versions of dependencies which have patched known security holes.

However, another core problem in this processing system is a lack of transparency within the codebase itself, making simple security audits and updates nearly impossible.
The system is also brittle, with a number of potential exploits currently prevented by minor variables or configurations -- if these were to change in an update the system could easily become far more vulnerable.
A redesign of this system could simplify the flow of data through it and make it far easier to understand and maintain, which would in turn make it far more secure.

\textbf{TODO wrap this up nicely, probably rewrite earlier bits as well}
