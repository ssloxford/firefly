\section{Motivation}

% https://directreadout.sci.gsfc.nasa.gov/?id=MODIS&date=2021-Nov-16

In recent decades, satellite imaging of the Earth's surface has become increasingly important within research and safety-critical contexts alike. 
Equipped with optical sensors that can detect a spectrum wider than visible light alone, satellites within NASA's \textit{Earth Observing System} (EOS) fleet have found broad usage within areas such as atmospheric observation, wildfire detection, and deforestation monitoring. \textbf{TODO: cite}
Automated data processing techniques are widely used to provide live data streams for incident response; ensuring data integrity at the radio downlink is therefore of high importance.

Although spoofing attacks have been addressed in recent satellite deployments through the introduction of cryptography, the long lifespans, high associated costs, and backwards compatibility requirements of existing satellite systems have resulted in a resistance towards retrofitting cryptography into these systems.
An additional desire for open data has motivated certain scientific satellite operators, including the EOS fleet operators, to make their data available through public unencrypted broadcast.
Accordingly, several prominent data processing software systems have been developed, which are run at ground stations across the world to receive and decode the transmitted signals, producing indispensable data for near real-time and retrospective Earth monitoring applications.
These processing algorithms are distributed in various ways, with the most popular distribution mechanism being the \textit{International Planetary Observation Processing Package} (IPOPP). %TODO: clarify/justify this claim
IPOPP contains the leading edge EOS satellite decoding algorithms, which have been developed over time by many teams, and are chained together to decode the raw bitstreams into usable data products.

Despite the unauthenticated nature of the data, the processing software found in IPOPP is not built with safety or security in mind -- all received signals are passed through a complex software pipeline designed to decode, process, and store the data to make it useful for later work.
Should an attacker successfully inject their own signal it would be processed by this same system and pass through a wide range of software components, each with their own unique attack surface.

Therefore, through carefully crafting input data, an attacker can target the services which are responsible for decoding each part of the protocol.
For example, the near real-time forest fire alerting services can be targeted through injecting arbitrary image data, causing ficticious alerts or masking legitimate ones.
Malicious packets can also be constructed at the protocol level, which lead to denial of service and arbitrary code execution when processed.
The attack data would be stored in various medium- to long-term storage public storage systems and potentially reprocessed at a later time.

Fixing these security issues is nontrivial thanks to innate constraints of the satellite hardware, and the complexity of its surrounding software ecoystem.
The attacker has multiple avenues through which to inject data, including through overshadowing the radio signal at receiver stations, and through posting malicious data on the satellite data mailing lists.
It is difficult to fully know the extent to which the system is vulnerable, thanks in part to its distribution mechanism, bundling large collections of potentially out-of-date software components with active CVEs, making a full audit challenging.

However, even though achieving theoretical guarantees of data authenticity is impossible without cryptographic primitives, the addition of simple countermeasures and renewed software engineering practises would serve to significantly increase the difficulty of achieving these attacks in a practical setting.
Firefly therefore validates existing work into satellite radio authenticity focussing on timing, waveform, and protocol-level analysis.

\subsection{Contributions}

In this paper we demonstrate that satellite data processing systems cannot assume that the input data is benign, and showcase the real world dangers associated with this assumption in currently deployed systems.

We demonstrate that, through overshadowing the wireless channel with carefully constructed packet data, an attacker can inject arbitrary bytes into targeted processing stages within a decoding pipeline.
This injection provides a gateway to more sophisticated attacks against the specific software implementation of the decoding system.

Against NASA's EOS flagship IPOPP, we achieve granular manipulation of the decoded images, denial of service in the near real-time context, and arbitrary code execution on end-user machines during dataset reanalysis.
Both near real-time processing systems and medium- to long-term storage archives are targeted by the malicious payloads, opening up both central systems and end user machines to attack.

% Engineering problems?

We go on to measure the feasibility of these attacks in a real-world setting, taking into account the particular difficulties associated with overshadowing the physical layer encoding phase shift keying in Section~\ref{sec:evaluation}.

We discuss countermeasures in Section~\ref{sec:countermeasures}, considering how authenticity of the decoded signals can be established within the constraints of EOS and similar satellite systems where cryptographic primitives are impractical and undesirable.
We show how even simple countermeasures based on timing, waveform, and protocol-level analysis can significantly increase the difficulty of achieving these attacks in a practical context.
Through generalising our approach, we propose scalable countermeasures to permit gradual adoption according to an organisation's tolerance for risk.

Finally, we discuss how practical outworkings of these new risks should affect the security posture within organisations, encompassing a discussion of untrusted user-submitted data and software architecture.
We pay attention to the software architectural principles required to make satellite decoding systems robust against this sort of attack, considering particular issues surrounding legacy software systems.
