\section{Related Work}

% TODO: this essentially needs completely rewriting

% Related work specifically to signal injection causing bad effects
% So SDR injection for instance, and other wireless systems

Much existing work within spoofing and jamming satellite communications focuses on Global Navigation Satellite Systems (GNSS).
The authors of~\cite{wuSpoofing2020} provide a comprehensive review of known types of spoofing and jamming attacks against these satellites, resulting in incorrectly reported positions and denial of service respectively.
Spoofing attacks are carried out by a variety of methods including replay and forgery in order to fool the receiver into accepting illegitimate messages, and jamming is typically performed using high-power interference.
This work also includes a review of methods by which spoofing and jamming can be detected or mitigated.
These methods include consistency checking, measuring signal characteristics, measuring arrival times~\cite{jedermann2021orbit}, utilizing arrays of antennas, measuring angle of signal arrival, and other factors extracted through machine learning techniques~\cite{oligeri2020past}.

In contrast to GNSS systems' continuous broadcast of raw signals, the systems we are investigating are packet-based, implementing a more complex protocol stack.
We apply spoofing and jamming concepts (and their associated countermeasures) to this novel target; we demonstrate that signal spoofing is possible for satellites for which the receiving and broadcasting equipment was previously thought to make such attacks prohibitively expensive, and show that attacks can have a significant effect on processing systems downstream from the original data.
We also look beyond the payload, analyzing the data processing system itself to create novel attacks.
In Section~\ref{sec:countermeasures} we explore how an existing system might apply known countermeasures, or adapt countermeasures from related fields, to reduce the attack surface of existing satellite systems.

There is also work in satellite security beyond GNSS -- in 2020 it was demonstrated that confidential maritime satellite communications could be reveived by SDR-equipped attackers from a great distance away (covering a total area of tens of millions of square kilometers), thanks to the satellites' wide beamwidth and unencrypted payload~\cite{pavurTale2020}.
This work also explores the theoretical possibilities of TCP session hijacking by broadcasting signals over a high-speed wired connection.
Our work differs in this respect -- we focus on an attacker capable of injecting signals directly over the radio link, compromising trust in the received signal.

Finally, we consider work in physical-layer radio security beyond the scope of satellite communication.
There are a number of deployed radio systems which do not employ cryptography, including the ``ADS-B'' protocol used to locate and track aircraft.
There is a good amount of research into the security of ADS-B: \cite{strohmeierSecurity2015} provides a good review of the extent to which the system is vulnerable and explores ways to make the system more secure through verifying the transmitter location or identity, or using cryptographic methods.
Once again, we can recontextualize a number of these techniques to provide improvements to the security of other unsecured systems, which we see in Section~\ref{sec:countermeasures}.

% To go in the attack description:
% \begin{comment}
  % The data is downlinked in almost exactly the same format for both the main data dump and the direct broadcast.
  % Packets from the MODIS instrument are encapsulated wthin the CCSDS Space Packet Protocol (SPP), which are packed within an unencrypted custom data link protocol known as the \textit{Channel Access Data Unit}, or CADU.
  % Finally, the CADUs are modulated using \textit{Quadrature Phase Shift Keying} (QPSK) and transmitted on the X-band, centered at 8160\,MHz. \textbf{TODO: is this the same for both DB and TDRSS dump?}
  %
  % % TODO: is such an in-depth explanation required here?
  % The raw SPP packet data is known as \textit{Level 0}, and is processed through a chain of programs distributed through the IPOPP framework to generate higher-level satellite-derived datasets.
  % The other EOS fleet satellites are processed in much the same way.
  % The Level 0 data is processed into \textit{Level 1}: an easier to use heirarchical data format, optionally geolocated to a subpixel accuracy using timing information, the satellite's orbital parameters, and an accurate model of the Earth's surface.
  % Level 1 data is processed into a variety of \textit{Level 2} datasets, including fire detection, land surface temperature values, vegetation detection, etc.
  % Finally, certain Level 3 datasets are produced, which generally consist of composites of the Level 2 data for specific purposes, such as analysing post-fire burned areas.
% \end{comment}
%
% % Tasked vs untasked images - aka when you have to ask to point the satellite, vs it just gets everything
% 
