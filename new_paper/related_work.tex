\section{Related Work}

Recent work into SATCOM security mostly considers eavesdroping, which the wide broadcast area of geostationary satellites makes possible.
For example, Pavur et. al. have demonstrated that high quantities of satellite internet traffic is conducted in plain HTTP through unencrypted satellite data links.  \cite{pavur2020tale}
Limited work into signal injection exists, but is limited to specific systems such as GNSS, with a limited discussion of countermeasures~\cite{tedeschi2022satellite}.
However, no current work seeks to understand the effects of a modern adversary spoofing non-GPS satellites through signal injection.

However, many existing satellites only partially implement encryption, or instead communicate in the clear; the standards required by government agencies are only beginning to require encryption in the space sector.
As a result, recent countermeasures have been proposed which analyze artifacts of overshadowing on the physical channel to determine the authenticity of the received data~\cite{jedermann2021orbit,oligeri2020past}.
However, the extent to which these countermeasures are effective in preventing overshadowing in a real world setup has yet to be determined.


Much existing work within spoofing and jamming satellite communications focuses on Global Navigation Satellite Systems (GNSS).
The authors of~\cite{wuSpoofing2020} provide a comprehensive review of known types of spoofing and jamming attacks against these satellites, resulting in incorrectly reported positions and denial of service respectively.
Spoofing attacks are carried out by a variety of methods including replay and forgery in order to fool the receiver into accepting illegitimate messages, and jamming is typically performed using high-power interference.
This work also includes a review of methods by which spoofing and jamming can be detected or mitigated.
These methods include consistency checking, measuring signal characteristics, measuring arrival times~\cite{jedermann2021orbit}, utilizing arrays of antennas, measuring angle of signal arrival, and other factors extracted through machine learning techniques~\cite{oligeri2020past}.

In contrast to GNSS systems' continuous broadcast of raw signals, the systems we are investigating are packet-based, implementing a more complex protocol stack.
We apply spoofing and jamming concepts (and their associated countermeasures) to this novel target; we demonstrate that signal spoofing is possible for satellites for which the receiving and broadcasting equipment was previously thought to make such attacks prohibitively expensive, and show that attacks can have a significant effect on processing systems downstream from the original data.
We also look beyond the payload, analyzing the data processing system itself to create novel attacks.
In Section~\ref{sec:countermeasures} we explore how an existing system might apply known countermeasures, or adapt countermeasures from related fields, to reduce the attack surface of existing satellite systems.

There is also work in satellite security beyond GNSS -- in 2020 it was demonstrated that confidential maritime satellite communications could be reveived by SDR-equipped attackers from a great distance away (covering a total area of tens of millions of square kilometers), thanks to the satellites' wide beamwidth and unencrypted payload~\cite{pavurTale2020}.
This work also explores the theoretical possibilities of TCP session hijacking by broadcasting signals over a high-speed wired connection.
Our work differs in this respect -- we focus on an attacker capable of injecting signals directly over the radio link, compromising trust in the received signal.

Finally, we consider work in physical-layer radio security beyond the scope of satellite communication.
There are a number of deployed radio systems which do not employ cryptography, including the ``ADS-B'' protocol used to locate and track aircraft.
There is a good amount of research into the security of ADS-B: \cite{strohmeierSecurity2015} provides a good review of the extent to which the system is vulnerable and explores ways to make the system more secure through verifying the transmitter location or identity, or using cryptographic methods.
Once again, we can recontextualize a number of these techniques to provide improvements to the security of other unsecured systems, which we see in Section~\ref{sec:countermeasures}.

\textbf{Countermeasures?}
