\section{Related Work}

% TODO: cite Manulis paper asking for a threat model
% TODO: write about information-theoretic noise model stuff
% TODO: write about GPS in particular, with its spoofing, fingerprinting, countermeasures - basically everything explored, but within this unique context

Eavesdropping is the focus of existing SATCOM security research, which shows that legacy protocols can often be decoded given the wide broadcast area and the lack of robust cryptography on the channel.
Existing research on signal injection exists, but so far only considers specific systems such as GPS.
Both topics draw upon existing wireless channel security research, applying these principles to understand the practicalities of attacking a specific system.

Accordingly, several countermeasures to these attack types have been proposed.
These primarily use physical-layer characteristics of the signal to provide additional secrecy and authenticity guarantees without requiring modification of the satellite.
We proceed to discuss each of these attacks, and proposed countermeasures, in turn.

\subsection{Eavesdropping}

% Basic point: encrypted downlinks are hardly universal
% Unencrypted for many reasons

% TODO: mention existing work in reversing Iridium

Fundamentally, securing communications over a wireless channel requires cryptographic guarantees about data confidentiality, integrity, and authenticity.
Satellite systems are no exception, especially considering the public nature of the channel; geostationary satellites can often be received across distances of thousands of kilometers.
As a result, unless the data link is robustly encrypted, anyone with access to suitable radio equipment can receive and decode these transmissions.

Pavur et. al. recently uncovered that since many satellite internet providers do not encrypt the downlink~\cite{pavur2020tale}, customer information is broadcast in the clear.
This allowed the decoding of ship crew passport information, aircraft in-flight system data, and POS terminal information.
This advanced previous work from 2005 which noted the huge amount of information sent via unsecured satellite broadcast~\cite{adelsbach2005satellite}.
This work also explores the theoretical possibilities of TCP session hijacking by broadcasting signals over a high-speed wired connection.

Cybercriminal gangs have been known to abuse the broadcast property to secretly exfiltrate data; packets sent in the clear to any known satellite customer IP address are receivable anywhere within the satellite's broadcast area~\cite{satellite_apt}.
Although newer satellite internet providers such as Starlink address this issue, and government agencies are beginning to require encryption~\textbf{TODO: cite}, these legacy systems are still widely used due to long mission lifespans and backwards compatibility with existing receiver hardware.

Scientific data is often also broadcast in the clear; NASA's Earth Observing Systems satellites are unencrypted by design to allow reception by custom groundstations.
However, since the data is not cryptographically signed, authenticity cannot be verified, opening the door to signal injection attacks.
Downstream systems such as NASA's Fire Information and Resource Management System (FIRMS) are therefore at risk of displaying falsified data.

Additionally, certain satellites that were considered secure at launch are now vulnerable.
For example, the Korean satellite COMS-1 uses single DES encryption~\cite{lrit-key-dec}, which has led to customer keys being successfully extracted from satellite data.
Additionally, the GEO-KOMPSAT-2A satellite had its keys leaked on the Korea Meteorological Administration website, which to this day remain publicly available~\cite{xrit-rx}.

\subsection{Signal injection}

Recent advancements in software-defined radio (SDR) hardware have reduced the requirements for spoofing on the wireless channel, by allowing attackers to capture, modify, and transmit radio information.
With only a simple amplifier and antenna setup, it is easy to transmit data in the UHF and VHF bands.
Other than GPS, the majority of work into signal injection considers non-satellite wireless systems; we touch upon each in turn.

\subsubsection{GPS spoofing}

GPS is a particularly good candidate for spoofing due to its high impact and the ease of overshadowing the communications.
Only a cheap SDR, simple amplifier and antenna, and some free software~\cite{gps-sdr-sim} is required, due to a number of factors.
High gain isn't required because the legitimate signal is low power, and is received by an omnidirectional antenna.
The signal is circularly polarized, meaning that the attacker doesn't need to be in phase, as per linear polarisation.
Since the signal is of a sufficiently low frequency, a high frequency upconverter and amplifier is not required, and the signal is not absorbed by objects meaning that line-of-sight is not required.

As a result, GPS spoofing attacks are well researched, with well-known real-world requirements in single- and multi-receiver and multi-attacker contexts~\cite{tippenhauer2011requirements}.
It is well understood how physical layer attributes, such as signal timing and precision affects the success of the attack.
The authors of~\cite{wuSpoofing2020} provide a comprehensive review of known types of spoofing and jamming attacks against these satellites, resulting in incorrectly reported positions and denial of service respectively.
% TODO: more GPS spoofing papers

GPS also shared similarities with other satellite systems due to its non-cryptographic integrity checks.
Although not designed to provide perfect integrity, GPS locks reject large deviations from a "locked on" position; overcoming this lock therefore requires the attacker generate signals to maintain continuity.
This principle applies in particular to Earth observation data, which is expected to be sampled from a continuous input; systems therefore detect anomalies by looking for large deviations between samples.

%\subsubsection{Non-GPS satellite receivers}

% TODO: scan space attacks open database to check for others

%Outside of a GPS context, the documented signal injection attacks surround satellite television

%Max headroom and similar - TV satellite
%Same protocols as used for satellite internet, traditionally

\subsubsection{Other wireless systems}

Signal injection attacks against the receivers of other wireless systems are also well explored.
These include WiFi, wireless sensors, mobile internet~\cite{yang2019hiding,erni2021adaptover}, GNSS~\cite{tippenhauer2011requirements}, and even avionic systems~\cite{sathayeWireless2019}.

\textbf{TODO: get some advice on this and write it}

Finally, we consider work in physical-layer radio security beyond the scope of satellite communication.
There are a number of deployed radio systems which do not employ cryptography, including the ``ADS-B'' protocol used to locate and track aircraft.
There is a good amount of research into the security of ADS-B: \cite{strohmeierSecurity2015} provides a good review of the extent to which the system is vulnerable and explores ways to make the system more secure through verifying the transmitter location or identity, or using cryptographic methods.
Once again, we can recontextualize a number of these techniques to provide improvements to the security of other unsecured systems, which we see in Section~\ref{sec:countermeasures}.


\subsection{Countermeasures}

Although only cryptography can provide the required secrety, integrity, and authenticity guarantees, these countermeasures require modification of the satellite.
Therefore, recent countermeasures have been proposed which analyze artifacts of overshadowing on the physical channel to determine the authenticity of the received data~\cite{jedermann2021orbit,oligeri2020past}.
Proposed anti-spoofing countermeasures rely upon and check for consistency in arrival times, signal angle, and the unique fingerprint of of the transmitting hardware.
Anti-eavesdropping countermeasures are instead run at the satellite, attempting to increase signal-to-noise ratio at the intended receiver, whilst minimising it at other locations.

\subsubsection{Data secrecy through channel noise}

One proposed countermeasure, known as information-theoretic security, involves decreasing the signal-to-noise (SNR) ratio at the receiver's location below a threshold where decoding is possible.
Several approaches for this have been proposed which assume knowledge of the eavesdropper's location, including beamforming the transmission at the intended receiver and emitting noise everywhere except the intended location.

However, this approach doesn't provide a defence against spoofing.

\textbf{TODO: finish this section}

% Are there satellites with frequencies overlapping with cellular?

% \subsection{Securing satellite relay systems through frequency hopping}


%Polarisation hopping, frequency hopping
%Beamforming

%https://ieeexplore.ieee.org/abstract/document/6739367?casa_token=rwzKea1crRoAAAAA:JYUiN2CdSAzZhFu0jOHq7-d-rINtM_STo4aCzUZasej6ailTUNX4MdaisXcM8G9qvl3AKWk
%https://ieeexplore.ieee.org/abstract/document/6108297?casa_token=hUULRC9BgwgAAAAA:tRxrOhLGzuLjOEE8S7UbXJ_5K4t6QZ0nPRpqiLgmDs7-1okaoezMIb5-HmK-eeCTsL5gcds
%https://ieeexplore.ieee.org/abstract/document/8822950?casa_token=-vLYsKFhvxIAAAAA:P6Xa33aZ_U191d0PGlmW-JdmkoUPYSBQDZ6e2svoM-qonIfm1WLji_smDt-qW1FTc2J2XI8
%https://ieeexplore.ieee.org/document/6638190
%https://ieeexplore.ieee.org/stamp/stamp.jsp?arnumber=6772207&casa_token=NFj6STXZ6mIAAAAA:US_nlhjLLvxy878JkL2F4-JjrgKZoYSH3SpbqwHyFGxbLadVk2foaH4WmccvW_WUe06ZZpc

\subsubsection{Timing analysis}



\subsubsection{Fingerprinting}

These methods include consistency checking, measuring signal characteristics, measuring arrival times~\cite{jedermann2021orbit}, utilizing arrays of antennas, measuring angle of signal arrival, and other factors extracted through machine learning techniques~\cite{oligeri2020past}.
