\section{Conclusion}

We have demonstrated that signal injection against satellite downlink processing systems can have a significant effect, not just on the raw data, but also on services that depend upon the derived datasets.
Through an end-to-end analysis of the processing pipeline, we have shown that satellites in NASA's EOS fleet are vulnerable, resulting in real-world impacts against a widely-used forest fire detection service.
These attacks have been enabled by the availability of software-defined radios coinciding with shifting security practices towards securing the physical layer.

Newly-designed space systems should be designed with this in mind, building robust authenticity guarantees through cryptography into all communications, and taking care to handle input data as untrusted.
We have discussed the extent to which existing systems can still be secured without the use of cryptography by considering countermeasures based on signal and timing analysis.

There is scope for future work validating our overshadowing simulations against real-world receiver dishes, which could demonstrate the attack in practice.
A comprehensive review is also required of satellite systems vulnerable to this type of attack, considering the possible effects on downstream systems which depend on their data.
Finally, further research is needed into non-cryptographic means of detecting and preventing signal overshadowing attacks on satellite systems, in order to provide increased security for legacy systems.
